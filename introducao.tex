\chapter{Introdução}

\textit{Clusters} de imagens são agrupamentos de imagens digitais segundo determinadas
características de similaridade, em geral cores, formas ou textura; o mapeamento
deve ocorrer de tal forma que cada classe forneça, essencialmente, as mesmas
informações sobre as imagens que a compõem. As técnicas de \textit{clustering}
associadas a mecanismos de classificação podem fornecer um resumo conciso do
conteúdo das imagens, podendo ser utilizado para diferentes tarefas relacionadas
com a gestão de banco de dados de imagens. Entre os principais usos dos \textit{clusters}
de imagens são: catálogos de arte, banco de imagens fotográficas, diagnósticos
médicos, computação forense e classificação de informações geográficas e de
sensoriamento remoto.

Este trabalho tem o objetivo de descrever e demonstrar a viabilidade de um
algorítimo não supervisionado de \textit{clustering} de imagens baseado numa rede neural
artificial de Kohonen, utilizando como caracterizadores das imagens os
descritores de Hu. Para alcançar este objetivo, além de toda pesquisa teórica
acerca dos métodos utilizados, uma aplicação real foi desenvolvida para testar
a validade do algoritmo proposto.

\section{Organização}

Este trabalho esta organizado da seguinte maneira.

No capítulo 2 é feita uma descrição geral dos métodos utilizados, abordando os
principais conceitos acerca de redes neurais artificiais, em especial das redes
de Kohonen, dos descritores de Hu e das imagens digitais. No capítulo 3 é
formalizado o algoritmo de \textit{clutering} proposto, bem como a justificativa para a
utilização dos métodos descritos no capítulo 2; ainda no capítulo 3 os
resultados de um conjunto de teste são apresentados e discutidos. No capítulo
4 são extraídas as conclusões e apresentadas as perspectivas de
trabalhos futuros.

